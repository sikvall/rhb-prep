\chapter{Frekvenslistor}
\label{kap:frekvenslistor}

\section{Kanalbeteckningar}

Det finns inga egentliga nationellt eller internationellt gångbara kanalbenämningar som är heltäckande. Amatörradion har sina, jaktradion har andra. Ofta är Kanal 1 (K1) den lägsta frekvensen men inte alltid, ibland har nya kanaler tillkommit efter de andra, så är exempelvis läget på jaktfrekvenserna på 155 MHz-bandet.

Det säkraste sättet är att prata i frekvens. Även om man har programmerat sin radio med ett antal kanaler är det ändå när man pratar med varandra enklast att ange frekvens. I de flesta fall behöver man ange ned till kHz för att vara tillräckligt noggrann. För exempelvis friluftskanalen ''Jakt-femman'' på 156,000 MHz anger man nollorna för att tydliggöra vilken frekvens som avses.

Med FM som modulation kan man avvika med någon kHz utan problem. Det kommer fungera ungefär lika bra om du ställer in radion på 156,003 MHz. För andra sändningsmoder kan det däremot vara viktigare, ju mer smalbandig en sändningsmod är desto viktigare blir frekvensinställningen.

För samband som nyttjar enkelt sidband (SSB/ESB) är frekvensinställningen så kritisk att man på radion behöver kunna justera frekvensen mycket noga. Ibland finns en särskild ratt för detta som då kallas \textit{clarifier} eftersom det låter ''kalle anka'' om man bara ligger lite snett i förhållande till sändarens frekvens. SSB är det vanligaste kommunikationssättet hos radioamatörer på kortvåg vid telefoni. Det är mindre vanligt på PR-banden och används knappt alls på VHF/UHF.

I det här dokumentet betecknar vi kanaler med en förkortning och en vedertagen siffra. Jaktkanalerna har prefixet J, kortdistansradiokanalerna kallas KDR eller SRBR plus en vedertagen siffra, 446-MHz bandets PMR-kanaler heter just PMR plus en vedertagen siffra.

Övriga kanaler betraktas som fritidskanaler eller övrigt fria kanaler och betecknas med FRI och siffra. Beteckningen F används inom marin VHF för fiskeflottans särskilda frekvenser så den undviker vi.

\section{VHF-bandet}

\section{UHF-bandet}

