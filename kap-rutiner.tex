\chapter{Rutiner för radiosamband}

För att framgångsrikt använda sin radio i ett krisläge eller pressad situation behöver man öva under lugna förhållanden. För att öva vettigt vill vi rekommendera att man:

\begin{itemize}
	\item övar tillsammans med någon som bor i närheten 
	\item träffas i grupp och övar olika sambandsscenarier
	\item funderar över och beskriver vad man vill uppnå, exempelvis:
	\begin{itemize}
		\item samband i skog och mark mellan individer och små grupper
		\item samband vid förflyttning i fordonskolonn
		\item samband med grannländer eller längre bort
		\item delta i att bygga upp ett kommunikationsnät för telegramförmedling
	\end{itemize}
	\item funderar på om inte amatörradiocertifikat vore en tillgång
\end{itemize}

\section{Schemalagd sändning}

\section{Kryptering}

\section{Sändningsförfarande}

\section{Mottagning av meddelande}

\section{Nödtrafik}

\section{Radiodisciplin}
