\chapter{Rutiner för radiosamband}

För att framgångsrikt använda sin radio i ett krisläge eller pressad situation behöver man öva under lugna förhållanden. För att öva vettigt vill vi rekommendera att man:

\begin{itemize}
	\item övar tillsammans med någon som bor i närheten 
	\item träffas i grupp och övar olika sambandsscenarier
	\item funderar över och beskriver vad man vill uppnå, exempelvis:
	\begin{itemize}
		\item samband i skog och mark mellan individer och små grupper
		\item samband vid förflyttning i fordonskolonn
		\item samband med grannländer eller längre bort
		\item delta i att bygga upp ett kommunikationsnät för telegramförmedling
	\end{itemize}
	\item funderar på om inte amatörradiocertifikat vore en tillgång
\end{itemize}

\section{Schemalagd sändning}
\todo{Schemalagd sändning, sched}

\section{Kryptering}
\todo{Täcktabell eller motsvarande krypto}

\section{Sändningsförfarande}
\todo{Praktiska tips för att sända och ta mot}

\section{Nödtrafik}
\todo{Att tänka på i samband med nödtrafik}

\section{Radiodisciplin}
\todo{Mer disciplin}

Radiodisciplin är något som ofta debatteras i radiokretsar, eller snarare bristen på densamma. Radio är en delad resurs och det finns nästan alltid folk som lyssnar på vad du säger i radion. Antingen andra användare eller så finns det automatiska mottagarstationer och pejlar.

För att det skall vara trevligt att prata radio med varandra finns det några enkla regler som amatörradion har använt i många år för att få en så trivsam miljö som möjligt med varandra och det finns ingen anledning att inte anamma dessa enkla regler direkt.

\subsection{Stör inte annan trafik}

Pågående trafik har \textit{alltid} företräde, även om du har bestämt att du skall använda denna frekvensen med en motstation så om det finns pågående trafik så har du två val:

Antingen kan du be de som talar vänligt att låna frekvensen ett kort ögonblick för att meddela din motstation en ny frekvens. Eller så kan du fråga om de som redan ligger på frekvensen har lust att byta. Det anses dock oartigt att flytta på folk och det bör vara dig själv du flyttar på i första hand. En tredje väg är att vänta tills radiosamtalet är över och därefter ropa upp motstationen.

Innan du sänder på kortvåg är det kutym att fråga om frekvensen är ledig. Det görs i mindre utsträckning på VHF/UHF eftersom utbredningen är mer lokal än på kortvågen som kan nå runt hela planeten ibland. Det är ändå ett smart drag för att undvika störningar och lyssna först en stund innan du frågar om frekvensen är ledig, vänta en kort stund och ropa sedan upp motstationen.

\subsection{Undvik vissa ämnen}

Ämnen som folk har starka känslor för är onödiga att ta upp på radion. Härtil hör frågor om pengar och inkomster, åsikter om religion eller politik och liknande. Andra personer som lyssnar kan påverkas och det är smart att inte reta upp folk i onödan eller trampa på ömma tår. Håll samtalet från sådana saker.

\subsection{Prata inte om identifierbara personer}

Det är onödigt att röja information om andra människor på radion. Prata inte om andra människor annat än i ytterst allmäna ordalag så att det inte går att identifiera vem som avses. Undvik skvallersnacket så långt möjligt.

I en situation där det är problem i samhället och radiosamband har tagit över från andra sambandsmedel som telefon så är det ännu viktigare att man är försiktig om vem man pratar. Det kan ju finnas många anledningar till att en person inte vill bli röjd och det är därför bättre att vara på den försiktiga sidan.

\subsection{Gör luckor i samtalet}

Samtal som pågår längre än några minuter bör lämna luckor ifall någon annan vill använda frekvensen. Detta är enkelt genom att då och då avsluta sitt sändningspass med frasen "och om någon annan vill komma in lämnar vi en lucka" sedan väntar motstationen ca 5 s innan den sänder så har personer som ligger och väntar en möjlighet att komma in och exempelvis bidra med information, ropa upp en motstation och flytta den till ny frekvens eller liknande.

\subsection{Lagom längd på sändningspassen}

Allt för korta växlingar fram och tillbaka innebär också att man sällan lämnar luckor och därmed gör det svårt för andra att bryta in i samtalet om de måste kort låna frekvensen. Allt för långa sändningspass belastar batterier i radion, gör det också svårt för folk att bryta in och så vidare.

Vid telegramsändning är det lämpligt att bryta efter några meningar och kontrollera att motstationen uppfattat dem innan man fortsätter. Markera i telegrammet var du är så du inte tappar bort dig ifall någon kommer och bryter in.

\subsection{Agera inte polis på radion}

Om du hör någon som stör, beter sig olämpligt på annat vis eller liknande så strunta i dem och byt kanal. Att börja diskutera  med folk som har ett dåligt beteende leder oftast till verbala brottningsmatcher på radion eller missförstånd. Det är bättre att själv uppträda korrekt och visa med exempel hur man skall göra. Alla har kanske inte läst radiohandbok för preppers!

