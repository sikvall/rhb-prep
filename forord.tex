\begingroup
\setlength{\parskip}{1em plus 0.5em minus 0.2em}
\setlength{\parindent}{0pt}
\section*{Förord}

Det finns många som är intresserade av prepping i vårt land precis som andra länder. Många har insett att kommunikation är av den största vikt, inte minst vid en större händelse när ordinarie kommunikationsnät kan vara utslagna. Det räcker med ett tillräckligt långt strömavbrott för att det skall vara omöjligt att meddela sig över telefon.

Många har därmed förstått att egen radioutrustning är ett smart komplement för att kunna meddela sig och söka kontakt med andra vare sig man är i nöd eller behöver enklare hjälp.

I dag florerar det flera billiga enkla handapparater och i denna handbok skall vi fokusera på de enklare 2-bandsradio man kan köpa för en billig peng från asien eller på auktionssiter på nätet.

Vi kommer också att titta på några enkla antennkonstruktioner eller färdigköpta som förlänger räckvidden jämfört med handapparatens gummipinne. Detta blir nödvändigt om man skall nå längre.

För att nå riktigt långt behövs andra typer av kommunikationsutrustning. Då pratar vi mellanvåg (MF) eller kortvåg (HF) med flera olika band. Ett kapitel i handboken kommer ta upp detta och var man kan söka sig för att få mer information och ta ett amatörradiocertifikat vilket är nödvändigt.

Det viktigaste med den här skriften är att undvika vanliga missförstånd, att få en förståelse för vad man kan förvänta sig av sin radioutrustning samt att försöka skapa en kunskapsbas att bygga vidare på för att senare ta amatörradiocertifikat (sändaramatörer är i bland annat USA starkt involverade i katastrofsamband) och kunna hålla radiosamband med andra med viss radiodisciplin.

Täpp-Anders Sikvall

\today
\endgroup
\clearpage
