\chapter{Radiotrafik och handhavande}
\label{kap:radiotrafik}

\section{Teknik}

\subsection{CTCSS}

CTCSS står för \textit{Continuous Tone-Coded Squelch System}, ungefär kontinuerligt ton-styrd brusspärr. Det används när det finns risk för störningar från annan trafik i närheten på samma frekvens, när bakgrundsstörningarna är tidvis

\section{Bokstavering}

Vid svårare förbindelser utnyttjar man bokstaveringsalfabeten för viktiga uppgifter. Bokstaveringsalfabeten är standardiserade och man bör inte avvika från de standardiserade uttrycken. På svenska använder vi det svenska bokstaveringsalfabetet normalt, vid kommunikation på engelska eller inter-nordiskt där personer som talar danska och norska är med är det bättre använda ITU-alfabetet.

Det finns inget som hindrar att man använder ITU-alfabetet på svenska också. Så gör man på VHF-marin även om resten av konversationen sker på svenska.

\subsection{Svenska bokstaveringsalfabetet}

\todo{Svenska bokstaveringsalfabetet}

\subsection{Internationella ITU-bokstaveringsalfabetet}

\todo{Internationella bokstaveringsalfabetet}


\section{Handhavande av radiosändaren}

\subsection{Handhållen radiostation}

Håll antennan förhållandelsevis rakt up. Polarisationen av radiovågorna är av betydelse. Ibland kan man vicka på radion $\pm$45 grader och på så vis förbättra förhållandena. När du hittat ett bra läge stå still. \todo{Fylla på med mer information}

\subsection{Bilburen radiostation}

\todo{Tips för bilburen radio}

\subsection{Fast radiostation}

\todo{Tips för fast radiostation}

\section{Förkortningar och uttryck}

\section{Utväxlande av meddelande}

\subsection{Anrop}

\subsubsection{Anropssignal}

\subsection{Utbyte av meddelande}

\subsection{Avslut}

\section{Nödsamtal och viktiga meddelanden}
\todo{Nödsamtal bör vi trycka på noga hur man agerar i en händelsekedja}

\section{Programvara}
\todo{Beskriva programvara för programmering av radio (Chirp)}
\todo{Fixa filer för Chirp, CSV-format}
\todo{Loggprogam, är det nödvändigt?}
