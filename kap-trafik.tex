\chapter{Radiotrafik och handhavande}
\label{kap:radiotrafik}

\section{Teknik}

\subsection{Subtonsbrusspärr, CTCSS}

CTCSS står för \textit{Continuous Tone-Coded Squelch System}, ungefär kontinuerligt ton-styrd brusspärr. Det används när det finns risk för störningar från annan trafik i närheten på samma frekvens, när bakgrundsstörningarna är tidvis stora och av sådan karaktär att de öppnar brusspärren i onödan kan det upplevas som störande. I dessa fall hjälper det att aktivera CTCSS.

Det som egentligen händer är att vid sändning lägger radion ut en subton. Det finns ett antal sådana att välja på i de flesta apparater. Tonen är inte hörbar för den filtreras bort ur mottagaren. De flesta radioapparater har ett frekvensmässigt omfång på ca 300--2700 Hz eller däromkring. Subtonerna ligger alltså under de 300 Hz som är den lägre gränsen.

Subtonerna brukar numreras och på PMR446 finns en standardiserad numrering i dessa apparater. För att undvika förvirring är det bra om man anger frekvens och nummer eller bara frekvensen. De flesta mer professionella radioapparater kan programmeras med subton tillsammans med frekvens och lagras i radion. Behöver man ändra kan man manuellt ställa om subtonen innan man sänder.

Ibland kan man sätta olika subton för sändning och mottagning. Det är sällan som det är väldigt praktiskt att göra det men ibland kan det vara smart att höra all sändning men låta en viss subton gå med när man sänder ändå. Då kan mottagaren välja att ha öppen mottagare eller ha subton påslagen.


\todo{CTCSS inte klarskriven!}

\section{Störningar}
\todo{Info om störningar, olika typer, vad man kan göra mm}

\section{Bokstavering}

Vid svårare förbindelser utnyttjar man bokstaveringsalfabeten för viktiga uppgifter. Bokstaveringsalfabeten är standardiserade och man bör inte avvika från de standardiserade uttrycken. På svenska använder vi det svenska bokstaveringsalfabetet normalt, vid kommunikation på engelska eller inter-nordiskt där personer som talar danska och norska är med är det bättre använda ITU-alfabetet.

Det finns inget som hindrar att man använder ITU-alfabetet på svenska också. Så gör man på VHF-marin även om resten av konversationen sker på svenska.

När man bokstaverar talar man tydligt och långsammare än normalt. Är förbindelsen riktigt dålig kan man bokstavera flera gånger. Om man skall bokstavera på en riktigt dålig förbindelse bokstaverar man varje ord två gånger efter varandra. Det ökar möjligheterna att höra. Sedan växlar man efter varje ord och kontrollerar att det uppfattats av mottagaren som då kan läsa ordet som svar.

Ibland hör man operatörer som bokstaverar varje bokstav två gånger. Det är inte det bästa sättet eftersom fädningsdippar och störningar ofta gör att man tappar hela ordet. Bokstavera så rytmiskt som möjligt med korta pauser mellan varje bokstavering.

Exempel: Ordet ''kalmar'' skall bokstaveras på en dålig förbindelse:

--- kalle adam ludvig martin rudolf ... kalle adam ludvig martin rudolf --- uppfattat, kom

--- kalmar, det är uppfattat, kom

\subsection{Svenska bokstaveringsalfabetet}

Värt att notera i det svenska bokstaveringsalfabetet är att alla bokstävers bokstaveringar är tvåställiga och endast manliga namn till skillnad från till exempel ITU-alfabetet som blandar friskt. Det finns en tanke bakom det och det är att det skall vara lättare att höra på en dålig förbindelse.

\begin{longtable}{ll|ll}
	\textbf{Symbol} & \textbf{Bokstavering} & \textbf{Symbol} & \textbf{Bokstavering} \\ \hline
	\endhead
	A               & Adam                  & V               & Viktor                \\
	B               & Bertil                & W               & Wilhelm               \\
	C               & Cesar                 & X               & Xerxes                \\
	D               & David                 & Y               & Yngve                 \\
	E               & Erik                  & Z               & Zäta                  \\
	F               & Filip                 & Å               & Åke                   \\
	G               & Gustav                & Ä               & Ärlig                 \\
	H               & Helge                 & Ö               & Östen                 \\
	I               & Ivar                  & 1               & Ett                   \\
	J               & Johan                 & 2               & Tvåa                  \\
	K               & Kalle                 & 3               & Trea                  \\
	L               & Ludvig                & 4               & Fyra                  \\
	M               & Martin                & 5               & Femma                  \\
	N               & Niklas                & 6               & Sexa                  \\
	O               & Oskar                 & 7               & Sju                   \\
	P               & Petter                & 8               & Åtta                  \\
	Q               & Qvintus               & 9               & Nia                   \\
	R               & Rudolf                & 0               & Nolla                 \\
	S               & Sigurd                & .               & Punkt                 \\
	T               & Tore                  & ,               & Komma                 \\
	U               & Urban                 & /               & Snedstreck
\end{longtable}



\subsection{Internationella ITU-bokstaveringsalfabetet}

ITU-alfabetet används om inte båda operatörerna är svensktalande. I marina sammanhang som på marin VHF bokstaverar man \textit{alltid} med ITU-alfabetet även om båda radiooperatörerna är svensktalande.

Det finns risk att förväxla en del saker i ITU-alfabetet som inte är lika välkonstruerat som det svenska. Betrakta exempelvis bokstaveringen \textit{sierra} för bokstaven s och \textit{zero} för siffran noll. 

Några siffror uttalas lite lustigt också för att öka tydligheten. Siffrorna två, tre och fem i synnerhet. Detta kommer från den militära sidan där \textit{five} kan låta som \textit{fire} vilket kan ha oönskade konsekvenser.

\begin{longtable}{ll|ll}
	\textbf{Symbol} & \textbf{Bokstavering} & \textbf{Symbol} & \textbf{Bokstavering} \\ \hline
	\endhead
	A               & Alfa                  & V               & Victor                \\
	B               & Bravo                 & W               & Whiskey               \\
	C               & Charlie               & X               & X-ray                 \\
	D               & Delta                 & Y               & Yankee                \\
	E               & Echo                  & Z               & Zulu                  \\
	F               & Foxtrot               & Å               & Alfa-Alfa             \\
	G               & Golf                  & Ä               & Alfa-Echo             \\
	H               & Hotel                 & Ö               & Oscar-Echo            \\
	I               & India                 & 1               & One                   \\
	J               & juliet                & 2               & To                    \\
	K               & Kilo                  & 3               & Tri                   \\
	L               & Lima                  & 4               & Four                  \\
	M               & Mike                  & 5               & Fife                  \\
	N               & November              & 6               & Six                   \\
	O               & Oscar                 & 7               & Seven                 \\
	P               & Papa                  & 8               & Eight                 \\
	Q               & Quebec                & 9               & Nine                  \\
	R               & Romeo                 & 0               & Zero                  \\
	S               & Sierra                & .               & STOP                  \\
	T               & Tango                 & ,               & Decimal               \\
	U               & Uniform               & /               & Stroke
\end{longtable}


\section{Handhavande av radiosändaren}

\subsection{Handhållen radiostation}

Håll antennan förhållandelsevis rakt up. Polarisationen av radiovågorna är av betydelse. Ibland kan man vicka på radion $\pm$45 grader och på så vis förbättra förhållandena. När du hittat ett bra läge stå still. \todo{Fylla på med mer information}

\subsection{Bilburen radiostation}

\todo{Tips för bilburen radio}

\subsection{Fast radiostation}

\todo{Tips för fast radiostation}

\section{Förkortningar och uttryck}

\section{Utväxlande av meddelande}

\subsection{Anrop}

\subsubsection{Anropssignal}

\subsection{Utbyte av meddelande}

\subsection{Avslut}

\section{Nödsamtal och viktiga meddelanden}
\todo{Nödsamtal bör vi trycka på noga hur man agerar i en händelsekedja}

\section{Programvara}
\todo{Beskriva programvara för programmering av radio (Chirp)}
\todo{Fixa filer för Chirp, CSV-format}
\todo{Loggprogam, är det nödvändigt?}
