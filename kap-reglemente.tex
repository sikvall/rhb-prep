\chapter{Regler och förordningar}

Det kan låta trist att gå igenom detta men det är ändå ganska viktigt. Dels för att det är väldigt dumt om man sänder på frekvenser som inte är tillåtna och orsakar störningar, dels för att också förstå hur man skall ställa in sin radioapparat för att det skall vara kompatibelt med andra apparater.

Det finns många olika delar och jag har försökt redovisa det som gäller i den mån jag själv förstår regler och bestämmelser. Observera dock att du är alltid själv skyldig att hålla dig ajour med bestämmelserna som kommer från PTS.

För de ''fria frekvenserna'' som vi använder finns regelverket i Post- och Telestyrelsens undantagsföreskrifter \textit{Undantag från tillståndsplikt} som berör dessa frekvenser. Det gällande dokumentet finner du på PTS hemsida\footnote{\url{https://www.pts.se/sv/Bransch/Radio/Radiotillstand/Undantag-fran-tillstandsplikt/}}. Det är i detta dokument som man kan utläsa tillåten effekt, bärvågsfrekvenser för vissa kanaler och mycket annat som rör den svenska frekvensplanen om radiosystem som inte kräver särskilda tillstånd.

Även om en radio inte kräver särskilt tillstånd betyder det inte att man får använda den hur som helst. Frekvenserna som är upplåtna skall användas till det syfte och ändamål som är specificerat. Därför följer att jaktkanalerna på 155 MHz-bandet inte kan användas som ett privatradioband (med undantag för 156,000 MHz eller ''jaktfemman'' som är en allmän fritidskanal också). Där så är specificerad finns ibland flera användningsområden för samma frekvensband, då får man acceptera störningar från andra användare. 

\section{Amatörradiobanden}

För amatörradiobanden gäller särskilda bestämmelser. Du måste ha avlagt två godkända prov, ett som handlar om teknik och ett om bestämmelser. När båda är godkända får du en anropssignal tilldelas av SSA (Sveriges Sändaramatörer) som är personlig.

Vid användning av amatörradiofrekvenser skall denna anropssignal användas och i övrigt skall reglementet som gäller för amatörradio följas. 

Anledningen till dessa prov är att sändaramatörer får bruka långt högre effekter samt bygga egen utrustning och till och med ansluta den till elnät och det ställer krav på kunskaper. Normalt är den högsta tillåtna sändareffekten för en amatörsändare 1 kW (det finns restriktioner i vissa band eller delband) och det är en helt annan sorts sändare vi talar om jämfört med PMR eller PR.

Amatörradion kan också nytja band från långvåg upp till flera GHz. Det finns alltså ganska många frekvensband att nyttja och man får även lära sig hur man jobbar med de olika banden för att nå dit man vill dagtid och nattetid och hur årsväxlingar, solfläckar, polarsken, meteoriter mm kan påverka utbredningen.

Jag vill rekommendera alla med seriöst sändarintresse att besöka närmaste radioklubb för att lära dig mer om hur man kan ta ett certifikat. Utbildningsmaterial för studie själv eller i grupp kan beställas från SSA:s webshop.

\todo{Länk till mer information om amatörradiokurs}

På VHF har amatörradion sitt band på 144--146 MHz och på UHF ligger bandet på 432--438 MHz. Inom dessa band finns sedan uppdelningar som bestämmer hur varje del används exempelvis för telegrafi, simplextrafik, repeatertrafik eller satellittrafik. Amatörerna har egna satelliter för särskild trafik.

Lyssna gärna på amatörradion men sänd inte om du inte har amatörradiocertifikat. Det är dumt att reta upp de som kan radio bättre än de flesta i onödan eftersom det dessutom är straffbart och radioamatörer har både kunskaper och resurser att pejla och har historiskt även sett till att personer som stört lagförts.

\section{Privatradiobandet 27 MHz}

PR-bandet är nog ett av de äldsta banden som folk nyttjat i preppingsammanhang och många andra sammanhang. Bestämmelserna i dag innebär att man får sända AM och FM med 4 W maximal utstrålad effekt (ERP) samt SSB med 12 W ERP på 40 kanaler.

Kanaldelningen är 10 kHz i detta band och bärvågsfrekvenserna skall ligga enligt kanallistan, det är alltså inte tillåtet sända ''mellan kanalerna'' på detta band. För de exakta frekvenserna se frekvenslistan i kapitel \ref{kap:frekvenslistor}. 

Effekten som anges är ERP \textit{Effective Radiated Power} och är den effekt som utstrålas i den bästa riktningen från en halvvågsdipol. Andra antenner som ger antennvinst innebär att sändarens effekt måste sänkas med motsvarande.

I vissa länder kallas det \textit{citizens band, CB} som betyder ungefär ''medborgarnas band. CB-radio och PR-radio har visst överlapp men det skiljer mellan de svenska upplånga frekvenserna och exempelvis de amerikanska. Vissa länder i Europa har också mindre skillnader mellan vilka kanaler och frekvenser som gäller. Detta kan vara förvirrande och det är därför säkrast att alltid dubbelkolla innan en PR-sändare används i annat land än hemlandet. 

\todo{* Kallas även Citizens Band --- Infört, kolla om det duger}
\todo{* Notera att regler och frekvenser skiljer sej mellan olika länder, vilket skapar förvirring --- infört, kolla om det duger}

Dvs om du har en antenn som ger en antennvinst av 3 dBd (eller 5,15 dBi) så måste sändarens effekt reduceras med 3 dB, dvs till hälften.

Radion skall också vara godkänd samt avsedd för trafik på bandet. Det innebär att en amatörradio som klarar dessa frekvenser \textit{inte får användas} om den inte särskilt har godkänts för detta. Godkända apparater har CE-märkning.

\begin{table}[h]
\begin{tabular}{lrl}
	Sändningsslag     & Max effekt & Mäts hur           \\ \hline
	FM, Fasmodulering &        4 W & Bärvågseffekt, ERP \\
	AM                &        4 W & RMS, ERP           \\
	SSB               &       12 W & PEP, ERP
\end{tabular}
\end{table}

\section{31 MHz-bandet}

31 MHz-bandet kallas ofta för jaktradio, men får användas även för annat bruk.
Här finns 40 kanaler och tillåten uteffekt är 5 W ERP. Kanaldelningen är 10 kHz.

Bandet är specifikt för Sverige och apparaterna för detta band är relativt ovanliga och kostsamma.

Godkända apparater har CE-märkning.

\section{69 MHz-bandet}

Detta band har kallats ''det nya PR-bandet'' och det är ett helt klart intressant band. Det får numera användas för ospecifik landmobil radiotrafik på 8 stycken särskilda kanaler. Det är inte tillåtet att sända på andra frekvenser än just dessa.

Effekten är högre än på PR-bandet, här tillåts upp till 25 W ERP (alltså ustrålad effekt från en dipol). Sändningscykeln tillåts vara max 10\% av tiden, dvs man får lyssna och sända med förhållandet 9:1 vilket är normalt vid komradiotrafik.

På frekvensen 69,0125 MHz tillåts endast mobila sändare inom Västra Götaland och Halland. Detta innebär att inga sändare med antenner monterade i fasta anläggningar som byggnader eller master är tillåtna här utan enbart handhållen radio och bilmonterade antenner är tillåtet.

Sändaren skall vara avsedd för bandet och godkänd, dvs CE-märkt. Amatörradiosändare får inte användas på detta band.

\todo{Det finns också 10 kanaler till med 5 W --- Hänvisa till frekvenslistan?}

\section{150 MHz jakt- och jordbrukskanaler}

Dessa kanaler är avsedda för den som jobbar inom skogsbruk, jordbruk eller bedriver jakt och behöver samband. De får inte användas av allmänheten i övrigt egentligen med ett undantag, kanal 5 (156,000 MHz) är en allmän öppen kanal som får användas av alla.

Just denna kanal kallas ibland för \textit{slaskkanal} eftersom den är utsatt för en massa störningar som regel och i vissa bilar också helt obrukbar då det finns signaler som kommer från bilens motorstyrning och andra processorer som läcker på denna frekvens. Även vissa typer av mobiltelefonisändare (basstationer) alstrar störningar på denna frekvens varför den ofta kan vara kraftigt störd.

Frekvensen 156,000 kallas ibland för fritidskanalen. Det är lämpligt att använda CTCSS på denna frekvens då den är delad med andra för att minska störningsriskerna.\todo{Förklara CTCSS i dokumentet}

Modulationen skall vara FM, kanaldelningen är 25 kHz, maximal effekt är 5 W ERP. Fyra av frekvenserna får inte användas på inhemskt svenskt vatten eller territorialhav då de kan störa marin VHF. Dessa frekvenser är: 155,400; 155,425; 155,450; och 155,475. Respektera detta särskilt i närheten av våra stora inre sjöar som Mälaren, Vänern, Vättern, Hjälmaren osv.

Apparater för dessa frekvenser skal vara avsedda för frekvensbandet och godkända med CE-märkning. Amatörradiostationer får ej användas.

\todo{* Förklara varför 156.000 av tekniska skäl är oanvändbar i många apparater --- Förklarat kolla att det duger}

\section{169 MHz VHF-kanal för ospecificerad användning}

Det finns en frekvens på 169,3875 MHz som får användas med max 500 mW ERP och 25 kHz kanaldelning. Denna är relativt ny och högintressant eftersom det finns så få fria frekvenser på denna del av VHF-bandet. I princip är det bara denna och 156,000 MHz som får användas som fritidsfrekvenser.

Apparaten skall som vanligt vara CE-märkt och godkänd för att användas. Amatörradiostationer får inte användas på denna frekvens.

\section{VHF Marin}

Det marina VHF-bandet får endast utnyttjas av de som har tillstånd och certifikat samt en tilldelad anropssignal. Anropssignalen följer normalt med ett fartyg. För att få certifikat skall en kurs och ett prov genomföras.

Utbildningen som är mycket bra tar bland annat upp hur man agerar i räddningskedjan om något händer till sjöss. Frekvenserna är viktiga och det rekommenderas att lyssna på dem om man är nära havet eller någon av de stora insjöarna.

Kanal 16 - 156,800 MHz är anrops- och nödkanal. Det är inte tillåtet att sända på marina VHF-bandet om man inte har tillstånd och certifikat samt godkänd utrustning. Amatörradioapparater är inte godkända här.

Det finns dock tillfällen när det kan vara tillåtet att använda utrustning utan tillstånd och det är om det handlar om att rädda någon i nöd. Det brukar heta att ''nöd bryter lag'' och vid sådana händelser är alla till buds stående medel tillåtna.

\section{444 MHz-bandet kortdistansradio}

Kallas även: KDR444, KDR, SRBR, kortdistansradio, short range business radio

Här finns 8 kanaler som får användas för landmobil trafik. Effekten är begränsad till 2 W ERP, och kanaldelningen är 25 kHz (normal FM). Radioapparaterna skall vara CE-märkta och det finns färdigprogrammerade professionella radioapparater att köpa på flera ställen för just detta band. Bandet kallas ibland Short Range Business Radio (SRBR), men det får användas av alla i Sverige, även för fritidsbruk.

Här förekommer ganska mycket kommersiell trafik, till exempel byggplatser, installatörer, lagerhantering, säkerhetspersonal vid idrottsevenemang, samband vid tävlingar, internradio på hotell och mycket annat. Eftersom det är en delad resurs är det viktigt att man tolererar störningar och är beredd att byta till en annan kanal om den man tänkt använda redan är upptagen.

Det rekommenderas också att använda tonöppnad brusspärr\footnote{Kallas även för tonselektiv, tonsquelch, pilottonsöppnad mottagning, CTCSS} för att dels själv slippa höra annan traftik men också göra det möjligt för andra att filtrera bort din trafik. Tonspärren gör dock inte att flera kan prata samtidigt på kanalen, man skall komma ihåg att sänder någon annan är kanalen upptagen även om du inte hör sändningen på grund av brusspärren.

Detta band är inte ett internationellt band men det finns även i vårt grannland Norge, mer information om detta finns i fribruksforskriften från norska motsvarigheten till PTS\footnote{\url{https://lovdata.no/dokument/SF/forskrift/2012-01-19-77/}}.

\section{PMR 446-bandet}

PMR446 är ett relativt internationellt PMR-band. Här tillåts endast färdiga godkända apparater avsedda för bandet med fast monterad antenn som inte kan bytas. Yttre antenner får alltså inte anslutas och effekten är begränsad till 500 mW ERP. Smalbandig FM\footnote{Narrow band FM, nFM, NB-FM} skall användas då kanaldelningen är 12,5 kHz. \todo{Förklara bandbredder i radioläran}

PMR-bandet är vanligt förekommande och apparater kan köpas nästan var som helst från leksaksbutiker till enkla elektronikbutiker och över nätet. Apparaterna skall vara CE-märkta. På grund av att de flesta inbyggda antenner samt tillåten effekt är låg kan man inte räkna med jättestort täckningsområde på dessa apparater. I terräng kan man räkna med mellan 0,5--2 km ungefär som bäst. 

Det som gör bandet intressant vid en händelse är den stora tillgången på radioapparater. I en händelse går det lätt att få tag i apparater så alla kan prata med varandra inom ett begränsat område åtminstone. Många kan även drivas med enkla torrbatterier (ofta av typen AAA, R03) vilket är en fördel om det inte finns elektricitet i området.

Precis som för KDR444-bandet är det stor risk för störningar från andra användare. Därför rekommenderas användning av tonöppnad brusspärr\footnote{pilotton, CTCSS, tone squelch, subtone} för att göra det lättare att slippa höra andras användning samt göra det möjligt för andra att filtrera bort ditt eget tal.

